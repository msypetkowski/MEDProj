\documentclass[a4paper]{article}

\usepackage[a4paper,  margin=0.4in]{geometry}

\usepackage{graphicx}
\usepackage{float}
\usepackage{multicol}
\usepackage{hyperref}
\usepackage{longtable}



\usepackage[utf8]{inputenc}
\begin{document}


\title{Project for MED classes - Implementation and comparison of K-means and HAG clustering algorithms}

\author{Michał Sypetkowski}
\setlength\columnsep{0.375in}  \newlength\titlebox \setlength\titlebox{2.25in}
\twocolumn
\maketitle


\section{Introduction}

In this raport, we implement and compare k-means\footnote{\url{https://en.wikipedia.org/wiki/K-means_clustering}}
and HAG\footnote{\url{https://en.wikipedia.org/wiki/Hierarchical_clustering}}clustering algorithms.
We perform accurate measurement on 4 datasets.

\section{Implementations}

We implement standard k-means algorithm with 2 different initialization techniques:
\begin{itemize}
        \item mean-std -- randomized with means and standard deviations calculated column-vise.
        \item Forgy -- by taking randomly sampled examples from dataset as initial k-means values. 
\end{itemize}

We implement HAG in 2 different complexies:
$O(n^3)$ and $O(n^2log(n))$ (using priority queue).
For small datasets, $O(n^3)$ is faster in our implementation because of python overheads.

Experiments are implementad to run automatically using one command,
and simple configuration including list of algorithms, metrics, etc.

\section{Experiments}

We run experiments on 4 datasets:
\begin{itemize}
\item iris -- 150 samples, 5 columns, 3 classes
\item adult -- 1000 samples, 92 columns, 2 classes
\item ctg  -- subset of 1000 samples, 23 columns, 10 classes
\item cars -- 1000 samples, 22 columns, 4 classes
\end{itemize}
For each dataset, we transform nominal attributes of $n$ possible values into $n$ numeric columns with 0 and 1 values.
We subset ctg, adult, and cars dataset to 1000 samples, to reduce variance in results caused by different samples count.
We use Adjusted Rand Score\footnote{\url{https://en.wikipedia.org/wiki/Rand_index}}
and purity\footnote{\url{https://nlp.stanford.edu/IR-book/html/htmledition/evaluation-of-clustering-1.html}} as metrics.
We perform all experiments with different number of clusters (from 1 to 15),
and with 2 different scaling methods -- min-max and mean-std.


On iris dataset (figures \ref{fig:iris_rand} and \ref{fig:iris_purity},
both algorithms achieve comparable results.
K-means is slightly better in terms of ARS.

On adult dataset (figures \ref{fig:adult_rand} and \ref{fig:adult_purity},
scaling method significantly affects the result.
Min-max scaling enables better clustering because of large number of columns.

On cars (figures \ref{fig:cars_rand} and \ref{fig:cars_purity}
and ctg (figures \ref{fig:ctg_rand} and \ref{fig:ctg_purity} datasets,
k-means significantly outperforms HAG.



\section{Conclusions}

Basing on our experiments, we observe that in general 
K-means works better, especially for datasets with a high nuber of class values (>3), and
HAG may eventually achieve better results for many clusters,
there are multiple columns in dataset, and there is small number of classes (<=3).
Moreover, HAG is not useful for clustering large datasets due to it's computional complexity.


\onecolumn

\begin{figure}[!hbt]
    \centering
    \includegraphics[trim={0cm 0cm 0cm 0cm},clip,page=1,width=1.0\textwidth]{../plots.pdf}
    \caption[]{Adjusted rand Score on iris dataset.
    \label{fig:iris_rand}
    }
\end{figure}
\begin{figure}[!hbt]
    \centering
    \includegraphics[trim={0cm 0cm 0cm 0cm},clip,page=2,width=1.0\textwidth]{../plots.pdf}
    \caption[]{Purity on iris dataset.
    \label{fig:iris_purity}
    }
\end{figure}

\begin{figure}[!hbt]
    \centering
    \includegraphics[trim={0cm 0cm 0cm 0cm},clip,page=3,width=1.0\textwidth]{../plots.pdf}
    \caption[]{Adjusted rand Score on adult dataset.
    \label{fig:adult_rand}
    }
\end{figure}
\begin{figure}[!hbt]
    \centering
    \includegraphics[trim={0cm 0cm 0cm 0cm},clip,page=4,width=1.0\textwidth]{../plots.pdf}
    \caption[]{Purity on adult dataset.
    \label{fig:adult_purity}
    }
\end{figure}

\begin{figure}[!hbt]
    \centering
    \includegraphics[trim={0cm 0cm 0cm 0cm},clip,page=5,width=1.0\textwidth]{../plots.pdf}
    \caption[]{Adjusted rand Score on ctg dataset.
    \label{fig:ctg_rand}
    }
\end{figure}
\begin{figure}[!hbt]
    \centering
    \includegraphics[trim={0cm 0cm 0cm 0cm},clip,page=6,width=1.0\textwidth]{../plots.pdf}
    \caption[]{Purity on ctg dataset.
    \label{fig:ctg_purity}
    }
\end{figure}

\begin{figure}[!hbt]
    \centering
    \includegraphics[trim={0cm 0cm 0cm 0cm},clip,page=7,width=1.0\textwidth]{../plots.pdf}
    \caption[]{Adjusted rand Score on cars dataset.
    \label{fig:cars_rand}
    }
\end{figure}
\begin{figure}[!hbt]
    \centering
    \includegraphics[trim={0cm 0cm 0cm 0cm},clip,page=8,width=1.0\textwidth]{../plots.pdf}
    \caption[]{Purity on cars dataset.
    \label{fig:cars_purity}
    }
\end{figure}
\end{document}
